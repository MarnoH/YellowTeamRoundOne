\documentclass[12pt]{article}
\usepackage{graphicx}
\usepackage{hyperref}

\begin{document}
	

	\begin{figure}
		\includegraphics[width=\linewidth]{logo.jpg}	
	\end{figure}

	\title 	{
				COS301\\
				Mini Project\\
				Functional Requirements Specification
		   	}
	\author {
				Duart Breedt u15054692\\
				Jordan Daubinet u15260870\\
				Marthinus Hermann u15081479\\
				Kgabo Moloto u13083865\\
				Rikard Schouwstra u15012299\\
				Jacques Smulders u15003087\\
				Mandla Mhlongo u29630135\
			}
	\maketitle
	\begin{center}
			\url{https://github.com/MarnoH/YellowTeamRoundOne}	
	\end{center}
	\newpage
	\tableofcontents
	\newpage
	\section{Introduction}
		
	\subsection{Purpose}
		The purpose of this document is to provide a detailed description of the requirements for the NavUP application. It will explain the purpose and features of the system, as well as system constraints.
This document is intended to be proposed to University of Pretoria lecturers and software development companies in mind, as well as a reference for developing the first version of the NavUP system.
		
	\subsection{Scope}
		NavUP is a mobile application that provides personnel on the University of Pretoria’s Hatfield campus the ability to conveniently navigate around the campus through the use of a range of features, technologies, and gamification principles.

NavUP will allow users to create optimal routes from their current location to a desired location. They will also be able to save previous routes and locations for easy access. NavUP will have the ability to send notifications to appropriate parties to inform them of achievements, events, and activities around campus. Login and register option will be available for students and staff to which their personal data will be saved.

 NavUP will also include ClickUP integration, class schedule management, and integration allowing users to ensure they can get to campus in a timely fashion. 
The real time nature of the crowdsourcing concept behind NavUP will make it accurate, responsive, and beneficial.
Users will be allowed to contribute to the crowdsourcing nature of NavUP by being able to report events such as protest action, and long queues at entrance gates or restaurants.

	\subsection{Definitions, Acronyms, and Abbreviations}
		{\bfseries Crowdsourcing} - The practice of obtaining information or input into a task or project by enlisting the services of a large number of people, either paid or unpaid, typically via the Internet.
		
		{\bfseries Gamification} - The application of typical elements of game playing (e.g. point scoring, competition with others, rules of play) to other areas of activity, typically as an online marketing technique to encourage engagement with a product or service.

		{\bfseries ClickUP} - A Blackboard Learn division dedicated to the University of Pretoria. It is a virtual learning environment and course management system developed by Blackboard Inc. It is Web-based server software which features course management, customizable open architecture, and scalable design that allows integration with student information systems and authentication protocols.
		
		{\bfseries User} - Someone who interacts with the NavUP application.

		{\bfseries UP} - University of Pretoria
		
		{\bfseries Student} - A student enrolled at the University of Pretoria
		
		{\bfseries Staff} - A staff member of the University of Pretoria
		
		{\bfseries Admin} - System Administrator who is given specific permission to manage and control the application.

		
	\subsection{References}
		Kung, D. (n.d.). Object-oriented software engineering. 1st ed.
		
	\subsection{Overview}
		{\bfseries Describe the rest of the SRS and its structure/layout}
	\section{Overall Description}	
	\subsection{Product Functions}
	\subsection{User Characteristics}
	\subsection{Constraints}
	- A connection to campus WI-FI is a constraint for NavUP. Since the application needs to locate the position of the user with enough accuracy to navigate through buildings and locate individual lecture halls, a constant internet connection (specifically campus WI-Fi) is required during the duration of the user’s walk between venues.
	- NavUP is also constrained by the level of accuracy required to navigate between locations within close proximity. Hence a comination of technologies needs to be implemented to meet this constraint.
	
	\subsection{Assumptions and Dependencies}
	\section{Specific Requirements}
	\subsection{Functional requirements}
	- Using NavUP, users will be able to search for a desired location. Depending on search criteria , the application will display the desired location on a map of campus, or generate an optimal route between the user's current and desired location. 
	- Students and Staff will be able to create custom timetables that will generate optimal routes between locations automatically.
	- Based on user interests, users can be notified of events (created by students and staff) happening on campus.
	- Users will be able to view leaderboards and earn achievements through the use of NavUP. 
	
	\subsection{Performance Requirements}
	- The system will be used by the students, guests and staff of up. If everyone uses the system then it should be able to handle +/- 50 000 users. \\
	- The application should provide accurate locations in a constantly changing environment. Heat maps, user preferences and suggestions should have a response time of a few seconds.\\
	- Maps of campus should be updated, to ensure accurate location estimation.
	\subsection{Design Constraints}
	- The Nav UP application must be able to run on a cellphone which has limited process power and battery life. It thus has to be efficient and not drain battery life quickly. \\
	- The application should not use a lot of bandwidth.\\
	- The interface should be mobile compatible.\\
	- Indoor navigation can only use wifi and not GPS.
	\subsection{Software System Attributes}
	- Users should have the option to withdraw all information gathered by the system.\\
	- The system should be available online as well as offline.\\
	- The system should stay updated, to ensure reliable information. For instance the maps of campuses should be updated regularly. 
	\section{Appendixes}
	\section{Index}

\end{document}
