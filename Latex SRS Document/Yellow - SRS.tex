\documentclass[12pt]{article}
\usepackage{graphicx}
\usepackage{hyperref}

\begin{document}
	

	\begin{figure}
		\includegraphics[width=\linewidth]{logo.jpg}	
	\end{figure}

	\title 	{
				COS301\\
				Mini Project\\
				Functional Requirements Specification
		   	}
	\author {
				Duart Breedt u15054692\\
				Jordan Daubinet u15260870\\
				Marthinus Hermann u15081479\\
				Kgabo Moloto u13083865\\
				Rikard Schouwstra u15012299\\
				Jacques Smulders u15003087\\
				Mandla Mhlongo u29630135\
			}
	\maketitle
	\begin{center}
			\url{https://github.com/MarnoH/YellowTeamRoundOne}	
	\end{center}
	\newpage
	\tableofcontents
	\newpage
	\section{Introduction}
		We have been approached by the University of Pretoria to develop an application namely NavUP. This application should give personnel which are on the University of Pretoria Hatfield campus the ability to conveniently navigate around the campus through the use of a range of features, technologies, and gamification principles.
		
	\subsection{Purpose}
		NavUP will provide students, staff, and guests with the ability to navigate from their current location to a desired location efficiently through the use of over 1000 Wi-Fi hotspots and crowd sourcing. This will allow personnel to avoid crowds and congestion between classes. The application will also give users access to features such as being notified of nearby events, profile management, and award based challenges among others.
		
	\subsection{Scope}
		NavUP will allow users to create routes from their current location to a desired location. They will also be able to save previous routes and locations for easy access. NavUP will have the ability to send notifications to appropriate parties to inform them of events, and activities around campus. Login and register option will be available for students and staff to which their personal data will be saved. NavUP will also have ClickUP integration and class schedule management and integration allowing users to ensure they can get to campus in a timely fashion. 
The real time nature of the crowdsourcing concept behind NavUP will make it accurate, responsive, and beneficial.
Users will be allowed to contribute to the crowdsourcing nature of NavUP by being able to report events such as protest action, and long queues at entrance gates or restaurants.

	\subsection{Definitions, Acronyms, and Abbreviations}
		{\bfseries Crowdsourcing} - The practice of obtaining information or input into a task or project by enlisting the services of a large number of people, either paid or unpaid, typically via the Internet.
		
		{\bfseries Gamification} - The application of typical elements of game playing (e.g. point scoring, competition with others, rules of play) to other areas of activity, typically as an online marketing technique to encourage engagement with a product or service.

		{\bfseries ClickUP} - A Blackboard Learn division dedicated to the University of Pretoria. It is a virtual learning environment and course management system developed by Blackboard Inc. It is Web-based server software which features course management, customizable open architecture, and scalable design that allows integration with student information systems and authentication protocols.
		
	\subsection{References}
		Kung, D. (n.d.). Object-oriented software engineering. 1st ed.
		
	\subsection{Overview}
		{\bfseries Describe the rest of the SRS and its structure/layout}
	\section{Overall Description}	
	\subsection{Product Functions}
	\subsection{User Characteristics}
	\subsection{Constraints}
	\subsection{Assumptions and Dependencies}
	\section{Specific Requirements}
	\subsection{Functional requirements}
	\subsection{Performance Requirements}
	
	\subsection{Design Constraints}
	\subsection{Software System Attributes}
	\section{Appendixes}
	\section{Index}

\end{document}
