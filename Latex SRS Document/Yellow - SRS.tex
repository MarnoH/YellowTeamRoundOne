\documentclass[12pt]{article}
\usepackage{graphicx}
\usepackage{hyperref}

\begin{document}
	

	\begin{figure}
		\includegraphics[width=\linewidth]{logo.jpg}	
	\end{figure}

	\title 	{
				COS301\\
				Mini Project\\
				Functional Requirements Specification
		   	}
	\author {
				Duart Breedt u15054692\\
				Jordan Daubinet u15260870\\
				Marthinus Hermann u15081479\\
				Kgabo Moloto u13083865\\
				Rikard Schouwstra u15012299\\
				Jacques Smulders u15003087\\
				Mandla Mhlongo u29630135\
			}
	\maketitle
	\begin{center}
			\url{https://github.com/MarnoH/YellowTeamRoundOne}	
	\end{center}
	\newpage
	\tableofcontents
	\newpage
	\section{Introduction}
		We have been approached by the University of Pretoria to develop an application namely NavUP. This application should give personnel which are on the University of Pretoria Hatfield campus the ability to conveniently navigate around the campus through the use of a range of features, technologies, and gamification principles.
		
	\subsection{Purpose}
		NavUP will provide students, staff, and guests with the ability to navigate from their current location to a desired location efficiently through the use of over 1000 Wi-Fi hotspots and crowd sourcing. This will allow personnel to avoid crowds and congestion between classes. The application will also give users access to features such as being notified of nearby events, profile management, and award based challenges among others.
		
	\subsection{Scope}
		NavUP will allow users to create routes from their current location to a desired location. They will also be able to save previous routes and locations for easy access. NavUP will have the ability to send notifications to appropriate parties to inform them of events, and activities around campus. Login and register option will be available for students and staff to which their personal data will be saved. NavUP will also have ClickUP integration and class schedule management and integration allowing users to ensure they can get to campus in a timely fashion. 
The real time nature of the crowdsourcing concept behind NavUP will make it accurate, responsive, and beneficial.
Users will be allowed to contribute to the crowdsourcing nature of NavUP by being able to report events such as protest action, and long queues at entrance gates or restaurants.

	\subsection{Definitions, Acronyms, and Abbreviations}
		{\bfseries Crowdsourcing} - The practice of obtaining information or input into a task or project by enlisting the services of a large number of people, either paid or unpaid, typically via the Internet.
		
		{\bfseries Gamification} - The application of typical elements of game playing (e.g. point scoring, competition with others, rules of play) to other areas of activity, typically as an online marketing technique to encourage engagement with a product or service.

		{\bfseries ClickUP} - A Blackboard Learn division dedicated to the University of Pretoria. It is a virtual learning environment and course management system developed by Blackboard Inc. It is Web-based server software which features course management, customizable open architecture, and scalable design that allows integration with student information systems and authentication protocols.
		
	\subsection{References}
		Kung, D. (n.d.). Object-oriented software engineering. 1st ed.
		
	\subsection{Overview}
		{\bfseries Describe the rest of the SRS and its structure/layout}
	\section{Overall Description}	
	- The following section will provide an overview of the whole NavUp application. The explanation of the application will inaugurate the basic functionality of it, as well as exhibiting how the application communicates with other systems. It will also describe the different types of collaborators that will make use of the application and also specify the different functionality that is available for each collaborator. At the end of this section we also present the constraints and assumptions that we've identified for the application. 
	\subsection{Product Functions}
	%With the NavUP application, users will be able to locate a venue on campus(including restrooms, cafes, bookshops etc) and determine the best possible route to the venue. The mobile version provides several search criteria enabling the administrator of the application to better manage the options for the different search criteria and provide accurate information according to that output. The results are displayed to the user screen in list view or map view(This could depend on the criteria included in the search). For buildings the list view will 
	
	\subsection{User Characteristics}
	There are four types of users that will interact with the application namely: Visitors, Students, Staff and Admin. Each of these users have their own requirements and they all have different use for the application. A Visitor can only search for a venue. He/She will most probably be using their mobile device. They must be able to search, select their desired locating from the resultant search and navigate to it. If perhaps the Visitor knows more about his/her desired destination advanced searches could also be allowed where the Visitor can then enter multiple criteria in order for the results to be more refined.
	
	The UP Staff will typically use the Web application where they'll manage information about consultation times, availability etc. They could for example provide their contact information and other details they see fit.
	
	The Administrators are assumed to only interact with the Web application for the are responsible for managing the overall application ensuring that the information the application retrieve for the user is always correct and up to date.
	\subsection{Constraints}
	\subsection{Assumptions and Dependencies}
	\section{Specific Requirements}
	\subsection{Functional requirements}
	\subsection{Performance Requirements}
	- The system will be used by the students, guests and staff of up. If everyone uses the system then it should be able to handle +/- 50 000 users. \\
	- The application should provide accurate locations in a constantly changing environment. Heat maps, user preferences and suggestions should have a response time of a few seconds.\\
	- Maps of campus should be updated, to ensure accurate location estimation.
	\subsection{Design Constraints}
	- The Nav UP application must be able to run on a cellphone which has limited process power and battery life. It thus has to be efficient and not drain battery life quickly. \\
	- The application should not use a lot of bandwidth.\\
	- The interface should be mobile compatible.\\
	- Indoor navigation can only use wifi and not GPS.
	\subsection{Software System Attributes}
	- Users should have the option to withdraw all information gathered by the system.\\
	- The system should be available online as well as offline.\\
	- The system should stay updated, to ensure reliable information. For instance the maps of campuses should be updated regularly. 
	\section{Appendixes}
	\section{Index}

\end{document}
